%%%%%%%%%%%%%%%%%%%%%%%%%%%%%%%%%%%%%%%%%%%%%
% Overleaf Template
% Authors: Dr. Veenstra and Jess Srinivas
% V2 with better boxes!
%%%%%%%%%%%%%%%%%%%%%%%%%%%%%%%%%%%%%%%%%%%%%

% CHANGE THESE DEFINITIONS

\newcommand{\NAME}{Max Ratcliff}
\newcommand{\ASSIGNMENT}{Assignment 3 -- Hangman Report}
\newcommand{\CLASS}{CSE 13S -- Spring 24}

% THEN WRITE YOUR PARAGRAPHS STARTING IN THE
% "Purpose" SECTION (AROUND LINE 44).

%%%%%%%%%%%%%%%%%%%%%%%%%%%%%%%%%%%%%%%%%%%%%

\documentclass{article}
\usepackage{graphicx} % Required for inserting images
\usepackage{hyperref}
\usepackage{lastpage}
\usepackage{fancyhdr}
\usepackage{geometry}
\geometry{margin=1in}
\usepackage{underscore}
\usepackage{subcaption}
\usepackage{fancyvrb}
\usepackage{listings}
\lstset{
basicstyle=\small\ttfamily,
columns=flexible,
breaklines=true
}


\title{\ASSIGNMENT}
\author{\NAME}
\date{\CLASS}

\begin{document}
\pagestyle{fancy}
\fancyfoot{}
\fancyhead{}
\fancyfoot[L]{\ASSIGNMENT\ -- \CLASS\ -- \NAME}
\fancyfoot[R]{\thepage}

\maketitle
k
%%%%%%%%%%%%%%%%%%%%%%%%%%%%%%%%%%%%%%%%%%%%%

% BEGIN WRITING YOUR DESIGN REPORT HERE

\section{Purpose}
\textbf{
The purpose of this program is to run a game of hangman for the user who will guess one letter at a time until the man is hung or the game is won.
}   
\section{Questions}
Please answer the following questions before you start coding. They will help guide you through the assignment. To make the grader's life easier, please do not remove the questions, and simply put your answers below the text of each question. 

To fill in the answers and edit this file, you can upload the provided zip file to overleaf by pressing [New project] and then [Upload project]. 

\subsection{Guesses}
One of the most common questions in the past has been the best way to keep track of which letters have already been guessed. To help you out with this, here are some questions (that you must answer) that may lead you to the best solution. 

\begin{itemize}
    \item How many valid single character guesses are there? What are they? \textbf{there are 26 valid character guesses representing all the lowercase letters a-z, and 6 guesses until the player loses}
    \item Do we need to keep track of how many times something is guessed? Do you need to keep track of the order in which the user makes guesses? \textbf{we do not need to keep track of how many times something is guessed as long as we keep track of what has been guessed since if its guessed more than one time we promt the user to redo their guess. we also dont need to track the order since the outputed eliminated guesses will be ordered alphabetically}
    \item What data type can we use to keep track of guesses? Keep in mind your answer to the previous questions. Make sure the data type you chose is easily printable in alphabetical order. \textbf{the best data type to keep track of the guesses would be a character array since its easy to itterate through and search and easy to represent as a string for printing. we can initiate a 6 character string and add a compare any new letters to the existing ones in the string to add it in alphabetical order}\footnote{Your answer should not involve rearranging the old guesses when a new one is made.}
    \item Based on your previous response, how can we check if a letter has already been guessed. \textbf{we can run a linear search on the string array which should only take a for loop}\footnote{The answer to this should be 1-2 lines of code. Keep it simple. If you struggle to do this, investigate different solutions to the previous questions that make this one easier.}
\end{itemize}

\subsection{Strings and characters}
\begin{itemize}
    \item Python has the functions \texttt{chr()} and \texttt{ord()}. Describe what these functions do. If you are not already familiar with these functions, do some research into them. 
    \item Below is some python code. Finish the C code below so it has the same effect. \footnote{Do not hard code any numeric values.}
    \begin{lstlisting}
        x = 'q'
        print(ord(x))
    \end{lstlisting}
    C Code:
    \begin{lstlisting}
        char x = 'q';
        print("%d", (int)x)
    \end{lstlisting}
    \item Without using \texttt{ctype.h} or \textbf{any} numeric values, write C code for \texttt{is_uppercase_letter()}. It should return false if the parameter is not uppercase, and true if it is. 
    \begin{lstlisting}
        #include <stdbool.h>
        char is_uppercase_letter(char x){
            if( (int)x < 65 || (int)x > 90){
                return false
            }
            return true
        }
    \end{lstlisting}
    
    \item What is a string in C? Based on that definition, give an example of something that you would assume is a string that C will not allow. \textbf{a string in C is an array of characters ending in a null terminator. "hello\textbackslash0world" is an example of a not allowed string as there is a null terminator in the middle}
    \item What does it mean for a string to be null terminated? Are strings null terminated by default? \textbf{it means that the last character is '\textbackslash0'. strings defined with " are automatically null terminated}
    \item What happens when a program that is looking for a null terminator and does not find it? \textbf{it runs off the edge of the char array resulting in an index out of bounds error}
    
\end{itemize}

\subsection{Testing}
List what you will do to test your code. Make sure this is comprehensive. \footnote{This question is a whole lot more vague than it has been the last few assignments. Continue to answer it with the same level of detail and thought.} Remember that you will not be getting a reference binary,
but you do have a file of inputs (\texttt{tester.txt}), and two output files (\texttt{expected_win.txt} and \texttt{expected_lose.txt}).
See the Testing section of the assignment PDF for how to use them. \textbf{I will make tests to test for the phrases 'zxywvutsrqponmlkjihgfedcba' and 'don't go in empty-handed' comparing to \texttt{expected_win.txt} and \texttt{expected_lose.txt}}
%\footnote{The output of your binary is not the only thing you should be testing!}.


\section{How to Use the Program}

\textbf{to use this program run \lstinline{./hangman "my secret phrase"} replacing what's in between the double quotes with the phrase you want to play with that is no longer than 256 characters, then the program will prompt you to enter one letter at a time until you either guess the secret phrase or fail to}

\section{Program Design}

\textbf{the main program is run in \lstinline{hangman.c}, and helper functions relating to character and string manipulation appear in \lstinline{hangman_helpers.c}}

\subsection{Overall Pseudocode}

\textbf{
\lstinline{hangman.c}
    \begin{itemize}
        \item get secret phrase
              loop until game is won
                print gallows
                print special characters
                check if phrase is only special characters
                    if so win game
                print eliminated characters
                get a letter guess from input
                check if letter appears in secret phrase
                    if so next iteration and reveal the guessed letter
                    if not add it to eliminated characters and increment gallows array index to print next frame
                check if phrase as been guessed
                    if so win game
                next loop iteration
    \end{itemize}}
\textbf{
\\
\lstinline{hangman_helpers.c}
    \begin{itemize}
        \item check if string contains character
                loop through index 0 to 256
                    check if character at index is desired character
                        if so return true
                    check if character at index is null terminator
                        if so return false
                return false
        \item read letter from input (and handle errors?)
                get sdtin
                check is 1 character
                    error
                check is valid character
                    error
                return letter
        \item check for lowercase letter
                make sure letter is lowercase by checking its in the ascii range from [97, 122]
        \item check validity of secret phrase
                iterate through potential secret from 0 to 256 or untill null-terminator found use helper method to check that the character is a lowercase letter.
                if character at index is not a lowercase letter
                    check cases of ' ' (ascii: 32), ''' (ascii: 44), '-' (ascii: 45)
                        if none of those cases print "invalid character: 'X'" where X is the invalid letter at the index
    \end{itemize}}

\subsection{Function Descriptions}
\textbf{
    \\
    bool string_contains_character()
\begin{itemize}
    \item Inputs: pointer to a string, character to find
    \item Outputs: true(1)/false(0)
    \item iterate through a string until target character is found, return true if found and false if not found
\end{itemize}}
\textbf{
    \\
    char read_letter()
\begin{itemize}
    \item Inputs: user input
    \item Outputs: character that user inputed
    \item get a character from the user, check for its validity
\end{itemize}}
\textbf{
    \\
    bool is_lowercase_letter()
\begin{itemize}
    \item Inputs: character
    \item Outputs: true(1)/false(0)
    \item make sure letter is lowercase by checking its in the ascii range from [97, 122]
\end{itemize}}
\textbf{
    \\
    bool is_valid_secret()
\begin{itemize}
    \item Inputs: string pointer to secret
    \item Outputs: true(1)/false(0)
    \item iterate through potential secret from 0 to 256 or untill null-terminator found use helper method to check that the character is a lowercase letter.
    if character at index is not a lowercase letter
        check cases of ' ' (ascii: 32), ''' (ascii: 44), '-' (ascii: 45)
            if none of those cases print "invalid character: 'X'" where X is the invalid letter at the index
\end{itemize}}
% Any references in your report appear at
% the end of the document automatically.
\bibliographystyle{unsrt}
\bibliography{bibtex}
\end{document}
