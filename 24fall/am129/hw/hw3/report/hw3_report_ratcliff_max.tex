\documentclass{article}
\usepackage{siunitx}
%% written with GPT prompted with:--------------------------
%% give me info on lstinline how do i set the code highlighting to fortran in line
\usepackage{listings}
\usepackage{xcolor}
\lstset{
    language=Fortran,           % Set the default language to Fortran
    basicstyle=\ttfamily,       % Use a monospaced font for the code
    keywordstyle=\color{blue},  % Set color for Fortran keywords
    commentstyle=\color{gray},  % Set color for comments
    stringstyle=\color{red},    % Set color for strings
    numbers=none,               % Disable line numbers for inline code
    breaklines=true,            % Automatically break long lines
}
%%--------------------------------------------------------
\begin{document}
    \section*{Implementation}
        This program solves a Boundary Value problem using the spectral collocation method
        over with a variable amount of discretization and then measures the error of the 
        estimation with the analytical solution, first we define the dft module which will let us perform a discrete fourier transform 
        on our function and then we create problemsetup to define our forcing function. Then we define the bvp and and use the tools outlines in the dtf module
        to solve the bvp. We then compare our solution to the provided analytical one and determine the error
    \section*{Questions}
        \subsection*{1.}
            in \texttt{bvp.f90} and \texttt{dft.f90} we do {\lstinline|use utility, only: fp, pi|}
            which enables us to use \texttt{real (fp)} throughout these files
        \subsection*{2.}
            \texttt{dft.f90} implements basic matrix multiplication and a discrete fourier transform so it 
            isnt super specific to this boundary value problem and could be used for others. \texttt{bvp.f90} uses
            the functions provided by \texttt{dft.f90} and then uses them to solve the specific BVP setup in 
            \texttt{problemsetup.f90}
        \subsection*{3.}
            because the size of arrays that we use are dependent on the level discretization. The more steps we take the bigger our 
            matrices are. They are allocated based on N in the \texttt{allocate\_data} subroutine and then at the end dealoccated by 
            \texttt{deallocate\_data}
        \subsection*{4.}
            elemental functions operate element wise on matrices allowing us to fill out our resulting array with the result at each point
            of discretization.
        \subsection*{5.}
            \begin{table}[ht]
                \centering
                \begin{tabular}{|c|c|}
                    \hline
                    \textbf{Amount of discretization} & \textbf{Error}\\
                    \hline
                    21 & \num{9.8179492425631842E-003} \\
                    41 & \num{5.5800848519638180E-007} \\
                    61 & \num{1.0906164860102763E-011} \\
                    81 & \num{6.2172489379008766E-015} \\
                    101 & \num{5.5511151231257827E-015} \\
                    \hline
                \end{tabular}
            \end{table}
            the error starts to decrease much much slower after 81 this indicates that we've reached a level of discretization small enough
            that we have a suitable estimation
        \subsection*{6.}
            with precision set to 6 decimal points \\
            \begin{table}[ht]
                \centering
                \begin{tabular}{|c|c|}
                    \hline
                    \textbf{Amount of discretization} & \textbf{Error}\\
                    \hline
                    21 & \num{9.82260704E-03} \\
                    41 & \num{8.10623169E-06} \\
                    61 & \num{2.38418579E-06} \\
                    81 & \num{4.52995300E-06} \\
                    101 & \num{3.81469727E-06} \\
                    \hline
                \end{tabular}
            \end{table}
            \\
            the error reaches its minimum value much faster and it is less precise
        \subsection*{7.}
            this assignment had us build the tools to perform a discrete fourier transform and then defines a forcing function 
            and then we define a function that we solve a boundary value problem for using the dft tools we created earlier
        \subsection*{8.}
            the real kind is 16 and the error is a much much longer number 

\end{document}